\section{Overview}
\subsection{Complete run}
In this section we make a full configuration with the administartions script ``setup.sh''. We describe every step. \newline First, we will install the ufw (uncomplicated firewall), which will then be configured by the script.
\begin{lstlisting}[escapeinside=||]
<INFO> - Tue Jan  8 11:14:31 UTC 2019 - No Modification Flag found. Seems to be the first run. Will start hardening now.
*** QUESTION *** Do you wish to perform a complete run (Firewall, DNS, SSH, Mail, Web)[y/n]?  |\colorbox{yellow}{y}|
<INFO> - Tue Jan  8 11:14:39 UTC 2019 - Complete run set to true

[...]

\end{lstlisting}
At the end of the whole configuration a modification flag is set, which is checked at a rerun. So you have the option modify and delete at a later time  (visible in the next section).
\begin{lstlisting}[escapeinside=||]

[...]

<INFO> - Tue Jan 8 11:25:19 UTC 2019 - Set |\colorbox{yellow}{modification Flag}|.
<INFO> - Tue Jan 8 11:25:19 UTC 2019 - Done. Finished with configurations

\end{lstlisting}

\newpage
\subsection{Rerun run}
If you run the script again at a later time, there are some small changes in the possibilities. New you will have the option ``Modify'', which makes it possible to configure all or certain components again (in the example only the firewall was configured again), or also the option ``Delete'', with which you could remove certain components.

\begin{lstlisting}[escapeinside=||]
*** QUESTION *** |\colorbox{yellow}{Modification Flag found}|. Please choose option: modify/uninstall [m/
u]?  |\colorbox{yellow}{m}|
<INFO> - Wed Jan  9 08:45:33 UTC 2019 - Modification choosen
*** QUESTION *** Do you wish to perform a complete run (Firewall, DNS, SSH, Mail, Web)[y/n]?  |\colorbox{yellow}{n}|
<INFO> - Wed Jan  9 08:45:34 UTC 2019 - Complete run set to false.
<INFO> - Wed Jan  9 08:45:34 UTC 2019 - Start the specific selection for single parts.
*** QUESTION *** Do you wish to perform action on fw  [y/n]?  |\colorbox{yellow}{y}|
<INFO> - Wed Jan  9 08:45:36 UTC 2019 - Action for fw set to true
*** QUESTION *** Do you wish to perform action on dns  [y/n]? |\colorbox{yellow}{n}|
<INFO> - Wed Jan  9 08:45:37 UTC 2019 - Action for dns set to false (will skip it).
*** QUESTION *** Do you wish to perform action on ssh  [y/n]?  |\colorbox{yellow}{n}|
<INFO> - Wed Jan  9 08:45:37 UTC 2019 - Action for ssh set to false (will skip it).
*** QUESTION *** Do you wish to perform action on mail  [y/n]?  |\colorbox{yellow}{n}|
<INFO> - Wed Jan  9 08:45:40 UTC 2019 - Action for mail set to false (will skip it).
*** QUESTION *** Do you wish to perform action on web  [y/n]?  |\colorbox{yellow}{n}|
<INFO> - Wed Jan  9 08:45:43 UTC 2019 - Action for web set to false (will skip it).

[...]

<INFO> - Wed Jan  9 08:45:55 UTC 2019 - Set modification Flag.
<INFO> - Wed Jan  9 08:45:55 UTC 2019 - Done. Finished with configurations
\end{lstlisting}

\subsubsection{Explanation of [...]} 
At this point specific components are configured, which are explained separately in this document. This section is only about the administration script, which triggers the whole processes.

\subsection{Overview process diagram}

\begin{figure}[H]
	\usetikzlibrary{shapes,arrows,calc}
	\centering
	\includestandalone[scale=0.6]{diagram/process_diagramm_setup_sh}
	\caption{Setup process diagram}
\end{figure}
\newpage
