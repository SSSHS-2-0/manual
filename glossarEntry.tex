
\newglossaryentry{ufw}
{
	name=ufw,
	description={The default firewall configuration tool for Ubuntu is ufw. Developed to ease iptables firewall configuration, ufw provides a user friendly way to create an IPv4 or IPv6 host-based firewall. By default UFW is disabled.\\SOURCE: \texttt{https://help.ubuntu.com/community/UFW}}
}
\newglossaryentry{DNS}
{
	name=DNS,
	description={The Domain Name System (DNS) is a hierarchical decentralized naming system for computers, services, or other resources connected to the Internet or a private network. It associates various information with domain names assigned to each of the participating entities. Most prominently, it translates more readily memorized domain names to the numerical IP addresses needed for locating and identifying computer services and devices with the underlying network protocols.\\SOURCE: \texttt{https://en.wikipedia.org/wiki/Domain\_Name\_System}}
}
\newglossaryentry{SMTP}
{
	name=SMTP,
	description={Simple Mail Transfer Protocol (SMTP) is an Internet standard for email transmission. First defined by RFC 821 in 1982, it was updated in 2008 with Extended SMTP additions by RFC 5321; which is the protocol in widespread use today.\\SOURCE: \texttt{https://en.wikipedia.org/wiki/Simple\_Mail\_Transfer\_Protocolm}}
}

\newglossaryentry{SSH}
{
	name=SSH,
	description={Secure Shell (SSH) is a cryptographic network protocol for operating network services securely over an unsecured network.[1] Typical applications include remote command-line login and remote command execution, but any network service can be secured with SSH. \\SOURCE: \texttt{https://en.wikipedia.org/wiki/Secure\_Shell}}
}

\newglossaryentry{wild-card}
{
	name=wild-card,
	description={In software, a wildcard character is a kind of placeholder represented by a single character, such as an asterisk (*), which can be interpreted as a number of literal characters or an empty string. It is often used in file searches so the full name need not be typed. \\SOURCE: \texttt{https://en.wikipedia.org/wiki/Wildcard\_character}}
}
\newglossaryentry{TLS}
{
	name=TLS,
	description={Transport Layer Security (TLS), and its now-deprecated predecessor, Secure Sockets Layer (SSL),[1] are cryptographic protocols designed to provide communications security over a computer network.[2] Several versions of the protocols find widespread use in applications such as web browsing, email, instant messaging, and voice over IP (VoIP). Websites can use TLS to secure all communications between their servers and web browsers. \\SOURCE: \texttt{https://en.wikipedia.org/wiki/Transport\_Layer\_Security}}
}

\newglossaryentry{SSL}
{
	name=SSL,
	description={Transport Layer Security (TLS), and its now-deprecated predecessor, Secure Sockets Layer (SSL),[1] are cryptographic protocols designed to provide communications security over a computer network.[2] Several versions of the protocols find widespread use in applications such as web browsing, email, instant messaging, and voice over IP (VoIP). Websites can use TLS to secure all communications between their servers and web browsers. \\SOURCE: \texttt{https://en.wikipedia.org/wiki/Transport\_Layer\_Security}}
}

\newglossaryentry{SPF}
{
	name=SPF,
	description={Sender Policy Framework (SPF) is an email authentication method designed to detect forged sender addresses in emails (email spoofing), a technique often used in phishing and email spam. \\SOURCE: \texttt{https://en.wikipedia.org/wiki/Sender\_Policy\_Framework}}
}

\newglossaryentry{DKIM}
{
	name=DKIM,
	description={DomainKeys Identified Mail (DKIM) is an email authentication method designed to detect forged sender addresses in emails, (email spoofing), a technique often used in phishing and email spam.  \\SOURCE: \texttt{https://en.wikipedia.org/wiki/DomainKeys\_Identified\_Mail}}
}

\newglossaryentry{DMARC}
{
	name=DMARC,
	description={DMARC (Domain-based Message Authentication, Reporting and Conformance) is an email-validation system designed to detect and prevent email spoofing, the use of forged sender addresses often used in phishing and email spam. . \\SOURCE: \texttt{https://en.wikipedia.org/wiki/DMARC}}
}
\newglossaryentry{IMAP}
{
	name=IMAP,
	description={In computing, the Internet Message Access Protocol (IMAP) is an Internet standard protocol used by email clients to retrieve email messages from a mail server over a TCP/IP connection.[1] IMAP is defined by RFC 3501.  \\SOURCE: \texttt{https://en.wikipedia.org/wiki/Internet\_Message\_Access\_Protocol}}
}


\newglossaryentry{DNSSEC}
{
	name=DNSSEC,
	description={The Domain Name System Security Extensions (DNSSEC) is a suite of Internet Engineering Task Force (IETF) specifications for securing certain kinds of information provided by the Domain Name System (DNS) as used on Internet Protocol (IP) networks. It is a set of extensions to DNS which provide to DNS clients (resolvers) origin authentication of DNS data, authenticated denial of existence, and data integrity, but not availability or confidentiality.\\SOURCE: \texttt{https://en.wikipedia.org/wiki/Domain\_Name\_System\_Security\_Extensions}}
}

\newglossaryentry{MTA-STS}
{
	name=MTA-STS,
	description={MTA-STS (full name SMTP Mail Transfer Agent Strict Transport Security) is a new standard that aims to improve the security of SMTP by enabling domain names to opt into strict transport layer security mode that requires authentication (valid public certificates) and encryption (TLS).\\SOURCE: \texttt{https://www.hardenize.com/blog/mta-sts}}
}


\newglossaryentry{WebRTC}
{
	name=WebRTC,
	description={WebRTC (Web Real-Time Communication) is a free, open-source project that provides web browsers and mobile applications with real-time communication (RTC) via simple application programming interfaces (APIs). It allows audio and video communication to work inside web pages by allowing direct peer-to-peer communication, eliminating the need to install plugins or download native apps.Supported by Google, Microsoft, Mozilla, and Opera, WebRTC is being standardized through the World Wide Web Consortium (W3C) and the Internet Engineering Task Force (IETF).\\SOURCE: \texttt{https://en.wikipedia.org/wiki/WebRTC}}
}
\newglossaryentry{Tor}
{
	name=Tor,
	description={Tor is free software for enabling anonymous communication. The name is derived from an acronym for the original software project name "The Onion Router". Tor directs Internet traffic through a free, worldwide, volunteer overlay network consisting of more than seven thousand relays to conceal a user's location and usage from anyone conducting network surveillance or traffic analysis.  \\SOURCE: \texttt{https://en.wikipedia.org/wiki/Tor\_(anonymity\_network)}}
}
\newglossaryentry{Docker}
{
	name=Docker,
	description={Docker is a computer program that performs operating-system-level virtualization, also known as ``containerization''. It was first released in 2013 and is developed by Docker, Inc. \\SOURCE: \texttt{https://en.wikipedia.org/wiki/Docker\_(software)}}
}
\newglossaryentry{DANE}
{
	name=DANE,
	description={DNS-based Authentication of Named Entities (DANE) is an Internet security protocol to allow X.509 digital certificates, commonly used for Transport Layer Security (TLS), to be bound to domain names using Domain Name System Security Extensions (DNSSEC) \\SOURCE: \texttt{https://en.wikipedia.org/wiki/DNS-based\_Authentication\_of\_Named\_Entities}}
}
\newglossaryentry{Ansible}
{
	name=Ansible,
	description={Ansible is open source software that automates software provisioning, configuration management, and application deployment. Ansible connects via SSH, remote PowerShell or via other remote APIs. . \\SOURCE: \texttt{https://en.wikipedia.org/wiki/Ansible\_(software)}}
}
\newglossaryentry{Python}
{
	name=Python,
	description={Python is an interpreted, high-level, general-purpose programming language. Created by Guido van Rossum and first released in 1991, Python has a design philosophy that emphasizes code readability, notably using significant whitespace. It provides constructs that enable clear programming on both small and large scales.\\SOURCE: \texttt{https://en.wikipedia.org/wiki/Python\_(programming\_language)}}
}
\newglossaryentry{Glue Records}
{
	name=Glue Records,
	description={Glue Records, or Nameserver Glue, relate a nameserver on the internet to an IP address. This relationship is set up at the domain registrar for the main domain on which the nameservers were created.\\SOURCE: \texttt{https://www.liquidweb.com/kb/what-are-glue-records/}}
}
\newglossaryentry{nsd}
{
	name=nsd,
	description={In Internet computing, NSD (for "name server daemon") is an open-source Domain Name System (DNS) server. It was developed by NLnet Labs of Amsterdam in cooperation with the RIPE NCC, from scratch as an authoritative name server (i.e., not implementing the recursive caching function by design).  \\SOURCE: \texttt{https://en.wikipedia.org/wiki/NSD}}
}
\newglossaryentry{unbound}
{
	name=unbound,
	description={Unbound is a validating, recursive, and caching DNS resolver product from NLnet Labs. It is distributed free of charge in open-source form under the BSD license. \\SOURCE: \texttt{https://en.wikipedia.org/wiki/Unbound\_(DNS\_server)}}
}
\newglossaryentry{Nginx}
{
	name=Nginx,
	description={Nginx  is a web server which can also be used as a reverse proxy, load balancer, mail proxy and HTTP cache. The software was created by Igor Sysoev and first publicly released in 2004.[9] A company of the same name was founded in 2011 to provide support and Nginx plus paid software. Nginx is free and open-source software, released under the terms of a BSD-like license. \\SOURCE: \texttt{https://en.wikipedia.org/wiki/Nginx}}
}
\newglossaryentry{Apache}
{
	name=Apache,
	description={The Apache HTTP Server, colloquially called Apache, is free and open-source cross-platform web server software, released under the terms of Apache License 2.0. Apache is developed and maintained by an open community of developers under the auspices of the Apache Software Foundation.  \\SOURCE: \texttt{https://en.wikipedia.org/wiki/Apache\_HTTP\_Server}}
}
\newglossaryentry{WAF}
{
	name=WAF,
	description={A web application firewall (or WAF) filters, monitors, and blocks HTTP traffic to and from a web application. A WAF is differentiated from a regular firewall in that a WAF is able to filter the content of specific web applications while regular firewalls serve as a safety gate between servers. By inspecting HTTP traffic, it can prevent attacks stemming from web application security flaws, such as SQL injection, cross-site scripting (XSS), file inclusion, and security misconfigurations.  \\SOURCE: \texttt{https://en.wikipedia.org/wiki/Web\_application\_firewall}}
}
\newglossaryentry{ModSecurity}
{
	name=ModSecurity,
	description={ModSecurity, sometimes called Modsec, is an open-source web application firewall (WAF). Originally designed as a module for the Apache HTTP Server, it has evolved to provide an array of Hypertext Transfer Protocol request and response filtering capabilities along with other security features across a number of different platforms including Apache HTTP Server, Microsoft IIS and NGINX. It is a free software released under the Apache license 2.0.   \\SOURCE: \texttt{https://en.wikipedia.org/wiki/ModSecurity}}
}
