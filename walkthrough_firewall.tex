\section{Firewall}
In this section we make a full Firewall configuration. We describe every step. \newline First, we will install the \gls{ufw} (uncomplicated firewall), which will then be configured by the script.
\begin{lstlisting}
<INFO> - Tue Jan  8 11:14:39 UTC 2019 - Starting Firewall Configurations.
<INFO> - Tue Jan  8 11:14:39 UTC 2019 - Will install 'ufw' now. Please wait...
..............
<INFO> - Tue Jan 8 11:14:52 UTC 2019 - Package 'ufw' is installed now.
\end{lstlisting}

After the successful installation it goes on with a basic security. This includes enabling all traffic out and blocking all traffic in.
So that nobody is locked out of his own server right at the beginning, seperat ssh on port 22 is enabled and configured as the only access from outside at this time.

\begin{lstlisting}[escapeinside=||]
<INFO> - Tue Jan  8 11:14:53 UTC 2019 - Ufw is enabled now.
<FW> - Tue Jan  8 11:14:53 UTC 2019 - UFW enable done.
<INFO> - Tue Jan  8 11:14:53 UTC 2019 - Start Firewall Hardening. (close all non relevant ports)
<INFO> - Tue Jan  8 11:14:54 UTC 2019 - All |\colorbox{yellow}{incoming}| and |\colorbox{yellow}{outgoing}| traffic is handeled now.
<FW> - Tue Jan  8 11:14:54 UTC 2019 - Traffic controll done.
<INFO> - Tue Jan 8 11:14:54 UTC 2019 - Activate |\colorbox{yellow}{SSH Connection}| for host 'XYZ'.
\end{lstlisting}

After setting up the base security, special configurations are loaded, which the user can add by himself. He does this by adding the necessary rules to the config-file ``fw.conf'' in the folder ``files''. The user has the possibility to say whether he wants to allow (ALLOW) or deny (DENY) a certain access.
Listed in the output are the minimum accesses needed for a comlete run of the scripts. These configurations are already present in the configuration file by default.
At the very end a list of the now activated rules will be displayed.

\begin{lstlisting}[escapeinside=||]
<INFO> - Tue Jan  8 11:14:55 UTC 2019 - Looking for Firewall Config file for specific configurations
<INFO> - Tue Jan  8 11:14:55 UTC 2019 - File Found. |\colorbox{yellow}{/root/files/fw.conf}|
# SSH
<INFO> - Tue Jan  8 11:15:19 UTC 2019 - Working on 'allow 22/tcp'.
<INFO> - Tue Jan  8 11:15:19 UTC 2019 - Working on 'allow 22/udp'.
# DNS
<INFO> - Tue Jan  8 11:15:20 UTC 2019 - Working on 'allow 53/tcp'.
<INFO> - Tue Jan  8 11:15:20 UTC 2019 - Working on 'allow 53/udp'.
# MAIL
<INFO> - Tue Jan  8 11:15:20 UTC 2019 - Working on 'allow 25/tcp'.
<INFO> - Tue Jan  8 11:15:20 UTC 2019 - Working on 'allow 25/udp'.
# SECURE SMTP
<INFO> - Tue Jan  8 11:15:21 UTC 2019 - Working on 'allow 465/tcp'.
<INFO> - Tue Jan  8 11:15:21 UTC 2019 - Working on 'allow 465/udp'.
# IMAP
<INFO> - Tue Jan  8 11:15:21 UTC 2019 - Working on 'allow 143/tcp'.
<INFO> - Tue Jan  8 11:15:21 UTC 2019 - Working on 'allow 143/udp'.
# IMAP TLS
<INFO> - Tue Jan  8 11:15:21 UTC 2019 - Working on 'allow 993/tcp'.
<INFO> - Tue Jan  8 11:15:22 UTC 2019 - Working on 'allow 993/udp'.
# HTTP HTTPS
<INFO> - Tue Jan  8 11:15:22 UTC 2019 - Working on 'allow 80/tcp'.
<INFO> - Tue Jan  8 11:15:22 UTC 2019 - Working on 'allow 443/tcp'.
<INFO> - Tue Jan  8 11:15:22 UTC 2019 - Done Specific configurations.
<FW> - Tue Jan  8 11:15:22 UTC 2019 - Specific Configurations of UFW done.
<INFO> - Tue Jan  8 11:15:22 UTC 2019 - Firewall Configurations done.
|\colorbox{yellow}{Status: active}|

To                         Action      From
--                         ------      ----
22/tcp                     ALLOW       Anywhere
22/udp                     ALLOW       Anywhere
53/tcp                     ALLOW       Anywhere
53/udp                     ALLOW       Anywhere
25/tcp                     ALLOW       Anywhere
25/udp                     ALLOW       Anywhere
465/tcp                    ALLOW       Anywhere
465/udp                    ALLOW       Anywhere
143/tcp                    ALLOW       Anywhere
143/udp                    ALLOW       Anywhere
993/tcp                    ALLOW       Anywhere
993/udp                    ALLOW       Anywhere
80/tcp                     ALLOW       Anywhere
443/tcp                    ALLOW       Anywhere
22/tcp (v6)                ALLOW       Anywhere (v6)
22/udp (v6)                ALLOW       Anywhere (v6)
53/tcp (v6)                ALLOW       Anywhere (v6)
53/udp (v6)                ALLOW       Anywhere (v6)
25/tcp (v6)                ALLOW       Anywhere (v6)
25/udp (v6)                ALLOW       Anywhere (v6)
465/tcp (v6)               ALLOW       Anywhere (v6)
465/udp (v6)               ALLOW       Anywhere (v6)
143/tcp (v6)               ALLOW       Anywhere (v6)
143/udp (v6)               ALLOW       Anywhere (v6)
993/tcp (v6)               ALLOW       Anywhere (v6)
993/udp (v6)               ALLOW       Anywhere (v6)
80/tcp (v6)                ALLOW       Anywhere (v6)
443/tcp (v6)               ALLOW       Anywhere (v6)

<FW> - Tue Jan  8 11:15:22 UTC 2019 - UFW Configurations done.
<FW> - Tue Jan 8 11:15:22 UTC 2019 - Actions on Firewall Done
\end{lstlisting}
\newpage

\subsection{Firewall process diagram}
Here we have process diagram of how the script works with all possible outcomes.

\begin{figure}[H]
	\usetikzlibrary{shapes,arrows,calc}
	\centering
	\includestandalone[scale=0.9]{diagram/process_diagramm_FW}
	\caption{Firewall process diagram}
\end{figure}
\newpage
