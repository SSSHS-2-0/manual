\section{Code directory tree}
\begin{table}[H]
    \scriptsize
\begin{minipage}{.5\textwidth}
\begin{Verbatim}[fontsize=\tiny,commandchars=\\\{\}]
.
├── dns
│   ├── dns.sh
│   ├── dns_nsd.sh
│   ├── dns_unbound.sh
│   ├── \colorbox{Apricot}{nsd}
│   │   ├── configBackwardsZoneDNS_nsd.sh
│   │   ├── configDNS_nsd.sh
│   │   ├── configForwardZoneDNS_nsd.sh
│   │   ├── finalisationDNS_nsd.sh
│   │   ├── installDNS_nsd.sh
│   │   └── testDNS_nsd.sh
│   ├── \colorbox{Apricot}{unbound}
│   │  ├── configDNSAccess_unbound.sh
│   │  ├── configDNSHardening_unbound.sh
│   │  ├── configDNSListening_unbound.sh
│   │  ├── finalisationDNS_unbound.sh
│   │  ├── installDNS_unbound.sh
│   │   └── testDNS_unbound.sh
│   └── uninstall_dns.sh
├── files
│   └── \colorbox{YellowGreen}{fw.conf -> fw/fw.conf}
├── fw
│   ├── controllTraffic.sh
│   ├── enableUfw.sh
│   ├── \colorbox{YellowGreen}{fw.conf}
│   ├── fw.sh
│   ├── specificConfigurations.sh
│   └── uninstall_fw.sh
├── mail
│   ├── alias.sh
│   ├── checkDomain.sh
│   ├── clientCertificate.sh
│   ├── \colorbox{SkyBlue}{dkim.sh}
│   ├── \colorbox{SkyBlue}{dmarc.sh}
│   ├── dnsRecords.sh
│   ├── dovecot.sh
│   ├── hardeningMail.txt
│   ├── mail.sh
│   ├── restart.sh
│   ├── \colorbox{SkyBlue}{spf.sh}
│   ├── tls.sh
│   └── uninstall_mail.sh
├── \colorbox{Rhodamine}{setup.sh}
├── ssh
│   ├── config.sh
│   ├── restart.sh
│   ├── ssh.sh
│   └── sshkeys.sh
├── utils
│   ├── checkPackage.sh
│   ├── chooseIp.sh
│   ├── getAllIpv4.sh
│   ├── getAllIpv6.sh
│   ├── getIpv4.sh
│   ├── getIpv6.sh
│   ├── logging.sh
│   ├── removeFolder.sh
│   ├── removePackage.sh
│   ├── revIpv4.sh
│   ├── summary.sh
│   ├── user.sh
│   └── valid_ipv4.sh
└── web
    ├── \colorbox{Goldenrod}{apache}
    │   ├── configureApache.sh
    │   └── enableApache.sh
    ├── \colorbox{Goldenrod}{nginx}
    │   ├── configureNginx.sh
    │   ├── enableNginx.sh
    │   └── nginxCertConfig.sh
    ├── uninstall_web.sh
    └── web.sh
\end{Verbatim}
\end{minipage}
\begin{minipage}{.5\textwidth}
    \begin{tabularx}{\textwidth}{|X|}
        \hline
        \cellcolor{Apricot} \textbf{DNS} \\
        \hline
        The DNS setup is based on two complety independent servers:
        \begin{itemize}
            \item{} \gls{nsd} as authoritative nameserver (queries from the internet to this domain).
            \item{} \gls{unbound} as local dns resolver (queries from this host).
        \end{itemize}
        % TODO
        % See section \ref{architecture_diagram}.
        \\
        \hline
        \cellcolor{YellowGreen} \textbf{Firewall} \\
        \hline
        The firewall configuration is loaded from this file (files/fw.conf). Standard ports are already defined, additional ports can be specified in this file. \\
        \hline
        \cellcolor{SkyBlue} \textbf{Anti-spam measures} \\
        \hline
        Following \gls{DNS} based anti-spam measures are configured for the mailserver. They makes sure spam mail is recognized during recieving and all sent mails, reach their destination without being classified as spam from the recieving side:
        \begin{itemize}
            \item{} \gls{DKIM}
            \item{} \gls{DMARC}
            \item{} \gls{SPF}
        \end{itemize} \\
        \hline
        \cellcolor{Rhodamine} \textbf{Entrypoint} \\
        \hline
        This is the main entrypoint for the setup (./setup.sh). From here on the user is guided through the whole setup process. \\
        \hline
        \cellcolor{Goldenrod} \textbf{Webserver} \\
        \hline
        As webserver two components interact together:
        \begin{itemize}
            \item{} \gls{Nginx} is used as a reverse proxy to terminate SSL connections and provide a secure HTTPS connection.
            \item{} \gls{Apache} is used as a web server to provide webpages, could later also be used as application server (see section \ref{future_work_web}).
        \end{itemize} \\
        % TODO
        % See section \ref{architecture_diagram}.
        \hline
    \end{tabularx}
\end{minipage}
\end{table}
